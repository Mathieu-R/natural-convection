% !TeX root = ./report.tex
\documentclass[a4paper]{tufte-handout}

%----------------------------------------------------------------------------------------
%	PREAMBLE
%----------------------------------------------------------------------------------------

% colored hyperlink
\hypersetup{colorlinks}

% book metadata (always in preamble)
\title{Convection naturelle: problème de Blasius}
\author{Mathieu Rousseau}
\date{\today}

% turn on numbering for parts and chapters
\setcounter{secnumdepth}{1}

\usepackage{silence}
% Filter warnings issued by package biblatex starting with "Patching footnotes failed"
\WarningFilter{biblatex}{Patching footnotes failed}

\usepackage{amsmath, amssymb}
\usepackage{setspace, enumerate, tikz}

% Non-stacked fractions and better unit spacing
\usepackage{units}

% typeset system of equations the easy way
% http://ftp.snt.utwente.nl/pub/software/tex/macros/generic/systeme/systeme_fr.pdf
\usepackage{systeme}

% add nabla, derivatives, bra-ket,... for physics
% http://mirrors.ibiblio.org/CTAN/macros/latex/contrib/physics/physics.pdf
\usepackage[arrowdel]{physics}

% https://tex.stackexchange.com/questions/167948/package-rerunfilecheck-warning-file-out-has-changed
\usepackage{bookmark}

% beautiful tables
\usepackage{booktabs}

% graphics / images
% \graphicspath{{}} = path of the img folder
\usepackage{graphicx}
\setkeys{Gin}{width=\linewidth,totalheight=\textheight,keepaspectratio}
\graphicspath{{figures/}}

% Default images settings
\setkeys{Gin}{width=\linewidth, totalheight=\textheight, keepaspectratio}

% to use a slightly smaller font
\usepackage{fancyvrb}
\fvset{fontsize=\normalsize}

% prints a trailing space in a smart way.
\usepackage{xspace}

% go to line the easy way
\usepackage[parfill]{parskip}

% units
\usepackage{siunitx}
\sisetup{locale = FR}

% increase vertical space for aligned equations
\setlength{\jot}{7pt}

% box around text
\usepackage{framed}

% cancel terms in equation
\usepackage[thicklines]{cancel}

\usepackage{color}

\definecolor{blue}{rgb}{0,0,1}
\definecolor{red}{rgb}{1,0,0}

% for theorem, proposition,...
% set a line break after the label
\usepackage[framed]{ntheorem}
\theoremstyle{break}

\newframedtheorem{defn}{Définition}%[section]
\newframedtheorem{prop}{Proposition}%[section]
\newtheorem{example}{Exemple}%[section]
\newtheorem{proof}{Preuve}

% conditions for equations
% https://tex.stackexchange.com/questions/95838/how-to-write-a-perfect-equation-parameters-description
\newenvironment{conditions}
  {\par\vspace{\abovedisplayskip}\noindent\begin{tabular}{>{$}l<{$} @{${}={}$} l}}
  {\end{tabular}\par\vspace{\belowdisplayskip}}

% generates an index
\usepackage{makeidx}
\makeindex

%----------------------------------------------------------------------------------------
%	COMMANDS
%----------------------------------------------------------------------------------------

% commands shortcuts
\newcommand{\mb}{\mathbb}
\newcommand{\R}{\mb{R}}
\newcommand{\dS}{\cdot d\vec{S}}

% prints an asterisk that takes up no horizontal space.
% useful in tabular environments.
\newcommand{\hangstar}{\makebox[0pt][l]{*}}

% Prints argument within hanging parentheses (i.e., parentheses that take
% up no horizontal space).  Useful in tabular environments.
\newcommand{\hangp}[1]{\makebox[0pt][r]{(}#1\makebox[0pt][l]{)}}

% cancel terms with color
\newcommand{\ccancel}[2]{\renewcommand{\CancelColor}{\color{#2}}\bcancel{#1}}

%----------------------------------------------------------------------------------------
%	DOCUMENT
%----------------------------------------------------------------------------------------

\begin{document}

\maketitle
\tableofcontents

\newpage

\section{Dérivation}

\noindent Nous allons analyser la convection naturelle le long d'une plaque chauffée verticale plane dont le bas se situe en $y=0$.

\begin{itemize}
  \item Le fluide colle à la plaque : $v = 0$ en $x = 0$.
  \item Le fluide est au repos loins de la plaque : $u = v = 0$.
  \item Le fluide est plus chaud proche de la plaque: $T_w > T_e$.
  \item $\delta << Y$
\end{itemize}

\noindent On suppose que l'écoulement n'existe que dans la couche limite ($\delta(x)$). \\ 
Dès lors on peut utiliser \textbf{les équations de Prandtl} en y ajoutant \textbf{le terme lié à la poussée d'Archimède} et en négligeant \textbf{la dissipation visqueuse}.

\noindent Hypothèses de Prandtl

\begin{itemize}
  \item L'écoulement est à grand nombre de Reynolds.
  \item L'épaisseur de la couche limite dépend du nombre de Reynolds $(\frac{\delta}{Y} = \frac{1}{\sqrt{Re}})$.
  \item Les forces d'inertie, de pression et de viscosité sont du même ordre de grandeur dans la couche limite.
\end{itemize}

\begin{itemize}
  \item Poussée d'Archimède (la pression ne dépend que de la profondeur) : $P(x, y) = \rho g y$
\end{itemize}

\noindent On peut obtenir une solution de similitude en définissant :
\begin{equation}
  \eta(x,y) = \frac{x}{\delta(y)} = \frac{x}{y} \left( \frac{Gr(y)}{4} \right)^{1/4}
\end{equation}

\begin{itemize}
  \item Nombre de Grasshof : $\frac{(\text{"poussée d'Archimède"})(\text{terme d'inertie})}{(\text{forces visqueuses})^2} = \frac{\beta \Delta T g L^3}{\nu^2}$
\end{itemize}

\begin{center}
  \begin{tikzpicture}
    % plaque chaude
    \draw (0,0) -- (0,5); node [below] {$y=0$}
    % couche limite
    \draw (0,0) parabola (2, 5);
    % profil de température
    \node[text=red] (-1,4.5) {$T_w$}
    \draw[red] (0,4.5) parabola (4,3);
    \node[text=red] (2,4) {$T_e$}
    % profil de vitesse
    \draw (-1,2) {v=0}
    \path (0,2) coordinate (p1)
    \path (2,2) coordinate (p2)
    \draw[blue] (p1) to [bend] (p2)
    \node (3,4) {$u = v = 0$}
    % axes
    \draw (-0.5, 0) <-Y-> (-0.5,4)
    \draw (4.5, 0) <-$\delta(y)$-> (4.5,3)
  \end{tikzpicture}
\end{center}

On part des équations de Navier-Stokes 2D :

\begin{align}
  \pdv{u}{x} + \pdv{v}{y} &= 0 \qq{(\textit{fluide incompressible})} \\
  \rho \left[ u \pdv{u}{x} + v \pdv{u}{y} \right] &= - \pdv{P}{x} + \mu \left[ \pdv[2]{u}{x} + \pdv[2]{u}{y} \right] \qq{(\textit{fluid flow})} \\
  \rho \left[ v \pdv{u}{x} + v \pdv{v}{y} \right] &= - \pdv{P}{y} + \mu \left[ \pdv[2]{u}{x} + \pdv[2]{u}{y} \right] \qq{(\textit{fluid flow})} \\
  u \pdv{T}{x} + v \pdv{T}{y} &= \alpha \left[\pdv[2]{T}{x} + \pdv[2]{T}{y} \right] \qq{(heat flow)}
\end{align}

%----------------------------------------------------------------------------------------
%	BIBLIOGRAPHY
%----------------------------------------------------------------------------------------

% Use \nobibliography{<bib file>}
% if you have references but don't want to print a bibliography
% Otherwise, use \bibliography{<bib file>}
\nobibliography{sample.bib}
\bibliographystyle{plainnat}

%----------------------------------------------------------------------------------------

\end{document}
