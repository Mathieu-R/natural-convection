\section{Dérivation}

\noindent Nous allons analyser la convection naturelle le long d'une plaque chauffée verticale plane dont le bas se situe en $y=0$.

\begin{itemize}
  \item Le fluide colle à la plaque : $v = 0$ en $x = 0$.
  \item Le fluide est au repos loins de la plaque : $u = v = 0$.
  \item Le fluide est plus chaud proche de la plaque: $T_w > T_e$.
  \item $\delta << Y$
\end{itemize}

\noindent On suppose que l'écoulement n'existe que dans la couche limite ($\delta(x)$). \\ 
Dès lors on peut utiliser \textbf{les équations de Prandtl} en y ajoutant \textbf{le terme lié à la poussée d'Archimède} et en négligeant \textbf{la dissipation visqueuse}.

\noindent Hypothèses de Prandtl

\begin{itemize}
  \item L'écoulement est à grand nombre de Reynolds.
  \item L'épaisseur de la couche limite dépend du nombre de Reynolds $(\frac{\delta}{Y} = \frac{1}{\sqrt{Re}})$.
  \item Les forces d'inertie, de pression et de viscosité sont du même ordre de grandeur dans la couche limite.
\end{itemize}

\begin{itemize}
  \item Poussée d'Archimède (la pression ne dépend que de la profondeur) : $P(x, y) = \rho g y$
\end{itemize}

\noindent On peut obtenir une solution de similitude en définissant :
\begin{equation}
  \eta(x,y) = \frac{x}{\delta(y)} = \frac{x}{y} \left( \frac{Gr(y)}{4} \right)^{1/4}
\end{equation}

\begin{itemize}
  \item Nombre de Grasshof : $\frac{(\text{"poussée d'Archimède"})(\text{terme d'inertie})}{(\text{forces visqueuses})^2} = \frac{\beta \Delta T g L^3}{\nu^2}$
\end{itemize}

\begin{center}
  \begin{tikzpicture}
    % plaque chaude
    \draw (0,0) -- (0,5); node [below] {$y=0$}
    % couche limite
    \draw (0,0) parabola (2, 5);
    % profil de température
    \node[text=red] (-1,4.5) {$T_w$}
    \draw[red] (0,4.5) parabola (4,3);
    \node[text=red] (2,4) {$T_e$}
    % profil de vitesse
    \draw (-1,2) {v=0}
    \path (0,2) coordinate (p1)
    \path (2,2) coordinate (p2)
    \draw[blue] (p1) to [bend] (p2)
    \node (3,4) {$u = v = 0$}
    % axes
    \draw (-0.5, 0) <-Y-> (-0.5,4)
    \draw (4.5, 0) <-$\delta(y)$-> (4.5,3)
  \end{tikzpicture}
\end{center}

On part des équations de Navier-Stokes 2D :

\begin{align}
  \pdv{u}{x} + \pdv{v}{y} &= 0 \qq{(\textit{fluide incompressible})} \\
  \rho \left[ u \pdv{u}{x} + v \pdv{u}{y} \right] &= - \pdv{P}{x} + \mu \left[ \pdv[2]{u}{x} + \pdv[2]{u}{y} \right] \qq{(\textit{fluid flow})} \\
  \rho \left[ v \pdv{u}{x} + v \pdv{v}{y} \right] &= - \pdv{P}{y} + \mu \left[ \pdv[2]{u}{x} + \pdv[2]{u}{y} \right] \qq{(\textit{fluid flow})} \\
  u \pdv{T}{x} + v \pdv{T}{y} &= \alpha \left[\pdv[2]{T}{x} + \pdv[2]{T}{y} \right] \qq{(heat flow)}
\end{align}